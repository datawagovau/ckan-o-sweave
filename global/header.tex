%------------------------------------------------------------------------------
% Legrand's Orange Book header
%------------------------------------------------------------------------------
%%%%%%%%%%%%%%%%%%%%%%%%%%%%%%%%%%%%%%%%%
% The Legrand Orange Book
% LaTeX Template
% Version 1.4 (12/4/14)
%
% This template has been downloaded from:
% http://www.LaTeXTemplates.com
%
% Original author: Mathias Legrand (legrand.mathias@gmail.com)
% Adapted by: Florian Mayer (Florian.Mayer@dpaw.wa.gov.au)
%
% License:
% CC BY-NC-SA 3.0 (http://creativecommons.org/licenses/by-nc-sa/3.0/)
%
% Compiling this template (Legrand's version)
% This template uses biber for its bibliography and makeindex for its index.
% When you first open the template, compile it from the command line with the
% commands below to make sure your LaTeX distribution is configured correctly:
%
% 1) pdflatex main
% 2) makeindex main.idx -s global/StyleInd.ist
% 3) biber main
% 4) pdflatex main x 2
%
% After this, when you wish to update the bibliography/index use the appropriate
% command above and make sure to compile with pdflatex several times
% afterwards to propagate your changes to the document.
%
% This template also uses a number of packages which may need to be
% updated to the newest versions for the template to compile. It is strongly
% recommended you update your LaTeX distribution if you have any
% compilation errors.
%
% Important note:
% Chapter heading images should have a 2:1 width:height ratio,
% e.g. 920px width and 460px height.
%
%%%%%%%%%%%%%%%%%%%%%%%%%%%%%%%%%%%%%%%%%

%----------------------------------------------------------------------------------------
%  PACKAGES AND OTHER DOCUMENT CONFIGURATIONS
%----------------------------------------------------------------------------------------

\documentclass[11pt,fleqn]{book} % Default font size and left-justified equations

% Page margins
\usepackage[top=3cm,bottom=3cm,left=3.2cm,right=3.2cm,headsep=10pt,a4paper]{geometry}

% Required for specifying colors by name
\usepackage[usenames,dvipsnames,svgnames,table]{xcolor}

% Define the orange color used for highlighting throughout the book
\definecolor{ocre}{RGB}{243,102,25}

% Font Settings
\usepackage{avant} % Use the Avantgarde font for headings
%\usepackage{times} % Use the Times font for headings
\renewcommand{\familydefault}{\sfdefault} % sans serif font as default

% Use the Adobe Times Roman as the default text font together with math symbols
% from the Symbol, Chancery and Computer Modern fonts
\usepackage{mathptmx}

\usepackage{microtype} % Slightly tweak font spacing for aesthetics
\usepackage[utf8]{inputenc} % Required for including letters with accents
% \DeclareUnicodeCharacter{B0}{\textdegree}
% \DeclareUnicodeCharacter{00A0}{ }
\usepackage[T1]{fontenc} % Use 8-bit encoding that has 256 glyphs

% Bibliography
\usepackage[
  style=alphabetic,
  sorting=nyt,
  sortcites=true,
  autopunct=true,
  autolang=hyphen,
  hyperref=true,
  abbreviate=false,
  backref=true,
  backend=biber]{biblatex}
\addbibresource{global/bibliography.bib} % BibTeX bibliography file
\defbibheading{bibempty}{}

% Index
\usepackage{calc} % used for spacing the index letter headings correctly
\usepackage{makeidx} % Required to make an index
\makeindex % Tells LaTeX to create the files required for indexing
%----------------------------------------------------------------------------------------
%  VARIOUS REQUIRED PACKAGES
%----------------------------------------------------------------------------------------

\usepackage{titlesec} % Allows customization of titles

\usepackage{graphicx} % Required for including pictures

% CKAN-o-Sweave: do not use, as it collides with path of images downloaded from CKAN
%\graphicspath{{img/}} % Specifies the directory where pictures are stored

\usepackage{lipsum} % Inserts dummy text

\usepackage{tikz} % Required for drawing custom shapes

\usepackage[english]{babel} % English language/hyphenation

\usepackage{enumitem} % Customize lists
\setlist{nolistsep} % Reduce spacing between bullet points and numbered lists

\usepackage{booktabs} % Required for nicer horizontal rules in tables

\usepackage{eso-pic} % Required for specifying an image background in the title page

\usepackage{xstring} % Conditionals
%----------------------------------------------------------------------------------------
%	MAIN TABLE OF CONTENTS
%----------------------------------------------------------------------------------------

\usepackage{titletoc} % Required for manipulating the table of contents
\setcounter{tocdepth}{4}
\contentsmargin{0cm} % Removes the default margin
% Chapter text styling
\titlecontents{chapter}[1.25cm] % Indentation
{\addvspace{15pt}\large\sffamily\bfseries} % Spacing and font options for chapters
{\color{ocre!60}\contentslabel[\Large\thecontentslabel]{1.25cm}\color{ocre}} % Chapter number
{}
{\color{ocre!60}\normalsize\sffamily\bfseries\;\titlerule*[.5pc]{.}\;\thecontentspage} % Page number
% Section text styling
\titlecontents{section}[1.25cm] % Indentation
{\addvspace{5pt}\sffamily\bfseries} % Spacing and font options for sections
{\contentslabel[\thecontentslabel]{1.25cm}} % Section number
{}
{\sffamily\hfill\color{black}\thecontentspage} % Page number
[]
% Subsection text styling
\titlecontents{subsection}[1.25cm] % Indentation
{\addvspace{1pt}\sffamily\small} % Spacing and font options for subsections
{\contentslabel[\thecontentslabel]{1.25cm}} % Subsection number
{}
{\sffamily\;\titlerule*[.5pc]{.}\;\thecontentspage} % Page number
[]

%----------------------------------------------------------------------------------------
%	MINI TABLE OF CONTENTS IN CHAPTER HEADS
%----------------------------------------------------------------------------------------

% Section text styling
\setcounter{tocdepth}{3}
\titlecontents{lsection}[0em] % Indendating
{\footnotesize\sffamily} % Font settings
{}
{}
{}

% Subsection text styling
\titlecontents{lsubsection}[.5em] % Indentation
{\normalfont\footnotesize\sffamily} % Font settings
{}
{}
{}

%----------------------------------------------------------------------------------------
%	PAGE HEADERS
%----------------------------------------------------------------------------------------

\usepackage{fancyhdr} % Required for header and footer configuration

\pagestyle{fancy}
\renewcommand{\chaptermark}[1]{\markboth{\sffamily\normalsize\bfseries\chaptername\ \thechapter.\ #1}{}} % Chapter text font settings
\renewcommand{\sectionmark}[1]{\markright{\sffamily\normalsize\thesection\hspace{5pt}#1}{}} % Section text font settings
\fancyhf{} \fancyhead[LE,RO]{\sffamily\normalsize\thepage} % Font setting for the page number in the header
\fancyhead[LO]{\rightmark} % Print the nearest section name on the left side of odd pages
\fancyhead[RE]{\leftmark} % Print the current chapter name on the right side of even pages
\renewcommand{\headrulewidth}{0.5pt} % Width of the rule under the header
\addtolength{\headheight}{2.5pt} % Increase the spacing around the header slightly
\renewcommand{\footrulewidth}{0pt} % Removes the rule in the footer
\fancypagestyle{plain}{\fancyhead{}\renewcommand{\headrulewidth}{0pt}} % Style for when a plain pagestyle is specified

% Removes the header from odd empty pages at the end of chapters
\makeatletter
\renewcommand{\cleardoublepage}{
\clearpage\ifodd\c@page\else
\hbox{}
\vspace*{\fill}
\thispagestyle{empty}
\newpage
\fi}

%----------------------------------------------------------------------------------------
%	THEOREM STYLES
%----------------------------------------------------------------------------------------

\usepackage{amsmath,amsfonts,amssymb,amsthm} % For math equations, theorems, symbols, etc

\newcommand{\intoo}[2]{\mathopen{]}#1\,;#2\mathclose{[}}
\newcommand{\ud}{\mathop{\mathrm{{}d}}\mathopen{}}
\newcommand{\intff}[2]{\mathopen{[}#1\,;#2\mathclose{]}}
\newtheorem{notation}{Notation}[chapter]

%%%%%%%%%%%%%%%%%%%%%%%%%%%%%%%%%%%%%%%%%%%%%%%%%%%%%%%%%%%%%%%%%%%%%%%%%%%
%%%%%%%%%%%%%%%%%%%% dedicated to boxed/framed environements %%%%%%%%%%%%%%
%%%%%%%%%%%%%%%%%%%%%%%%%%%%%%%%%%%%%%%%%%%%%%%%%%%%%%%%%%%%%%%%%%%%%%%%%%%
\newtheoremstyle{ocrenumbox}% % Theorem style name
{0pt}% Space above
{0pt}% Space below
{\normalfont}% % Body font
{}% Indent amount
{\small\bf\sffamily\color{ocre}}% % Theorem head font
{\;}% Punctuation after theorem head
{0.25em}% Space after theorem head
{\small\sffamily\color{ocre}\thmname{#1}\nobreakspace\thmnumber{\@ifnotempty{#1}{}\@upn{#2}}% Theorem text (e.g. Theorem 2.1)
\thmnote{\nobreakspace\the\thm@notefont\sffamily\bfseries\color{black}---\nobreakspace#3.}} % Optional theorem note
\renewcommand{\qedsymbol}{$\blacksquare$}% Optional qed square

\newtheoremstyle{blacknumex}% Theorem style name
{5pt}% Space above
{5pt}% Space below
{\normalfont}% Body font
{} % Indent amount
{\small\bf\sffamily}% Theorem head font
{\;}% Punctuation after theorem head
{0.25em}% Space after theorem head
{\small\sffamily{\tiny\ensuremath{\blacksquare}}\nobreakspace\thmname{#1}\nobreakspace\thmnumber{\@ifnotempty{#1}{}\@upn{#2}}% Theorem text (e.g. Theorem 2.1)
\thmnote{\nobreakspace\the\thm@notefont\sffamily\bfseries---\nobreakspace#3.}}% Optional theorem note

\newtheoremstyle{blacknumbox} % Theorem style name
{0pt}% Space above
{0pt}% Space below
{\normalfont}% Body font
{}% Indent amount
{\small\bf\sffamily}% Theorem head font
{\;}% Punctuation after theorem head
{0.25em}% Space after theorem head
{\small\sffamily\thmname{#1}\nobreakspace\thmnumber{\@ifnotempty{#1}{}\@upn{#2}}% Theorem text (e.g. Theorem 2.1)
\thmnote{\nobreakspace\the\thm@notefont\sffamily\bfseries---\nobreakspace#3.}}% Optional theorem note

%%%%%%%%%%%%%%%%%%%%%%%%%%%%%%%%%%%%%%%%%%%%%%%%%%%%%%%%%%%%%%%%%%%%%%%%%%%
%%%%%%%%%%%%% dedicated to non-boxed/non-framed environements %%%%%%%%%%%%%
%%%%%%%%%%%%%%%%%%%%%%%%%%%%%%%%%%%%%%%%%%%%%%%%%%%%%%%%%%%%%%%%%%%%%%%%%%%
\newtheoremstyle{ocrenum}% % Theorem style name
{5pt}% Space above
{5pt}% Space below
{\normalfont}% % Body font
{}% Indent amount
{\small\bf\sffamily\color{ocre}}% % Theorem head font
{\;}% Punctuation after theorem head
{0.25em}% Space after theorem head
{\small\sffamily\color{ocre}\thmname{#1}\nobreakspace\thmnumber{\@ifnotempty{#1}{}\@upn{#2}}% Theorem text (e.g. Theorem 2.1)
\thmnote{\nobreakspace\the\thm@notefont\sffamily\bfseries\color{black}---\nobreakspace#3.}} % Optional theorem note
\renewcommand{\qedsymbol}{$\blacksquare$}% Optional qed square
\makeatother

% Defines the theorem text style for each type of theorem to one of the three styles above
\newcounter{dummy}
\numberwithin{dummy}{section}
\theoremstyle{ocrenumbox}
\newtheorem{theoremeT}[dummy]{Theorem}
\newtheorem{problem}{Problem}[chapter]
\newtheorem{exerciseT}{Exercise}[chapter]
\theoremstyle{blacknumex}
\newtheorem{exampleT}{Example}[chapter]
\theoremstyle{blacknumbox}
\newtheorem{vocabulary}{Vocabulary}[chapter]
\newtheorem{definitionT}{Definition}[section]
\newtheorem{corollaryT}[dummy]{Corollary}
\theoremstyle{ocrenum}
\newtheorem{proposition}[dummy]{Proposition}

%----------------------------------------------------------------------------------------
%	DEFINITION OF COLORED BOXES
%----------------------------------------------------------------------------------------

\RequirePackage[framemethod=default]{mdframed} % Required for creating the theorem, definition, exercise and corollary boxes

% Theorem box
\newmdenv[skipabove=7pt,
skipbelow=7pt,
backgroundcolor=black!5,
linecolor=ocre,
innerleftmargin=5pt,
innerrightmargin=5pt,
innertopmargin=5pt,
leftmargin=0cm,
rightmargin=0cm,
innerbottommargin=5pt]{tBox}

% Exercise box
\newmdenv[skipabove=7pt,
skipbelow=7pt,
rightline=false,
leftline=true,
topline=false,
bottomline=false,
backgroundcolor=ocre!10,
linecolor=ocre,
innerleftmargin=5pt,
innerrightmargin=5pt,
innertopmargin=5pt,
innerbottommargin=5pt,
leftmargin=0cm,
rightmargin=0cm,
linewidth=4pt]{eBox}

% Definition box
\newmdenv[skipabove=7pt,
skipbelow=7pt,
rightline=false,
leftline=true,
topline=false,
bottomline=false,
linecolor=ocre,
innerleftmargin=5pt,
innerrightmargin=5pt,
innertopmargin=0pt,
leftmargin=0cm,
rightmargin=0cm,
linewidth=4pt,
innerbottommargin=0pt]{dBox}

% Corollary box
\newmdenv[skipabove=7pt,
skipbelow=7pt,
rightline=false,
leftline=true,
topline=false,
bottomline=false,
linecolor=gray,
backgroundcolor=black!5,
innerleftmargin=5pt,
innerrightmargin=5pt,
innertopmargin=5pt,
leftmargin=0cm,
rightmargin=0cm,
linewidth=4pt,
innerbottommargin=5pt]{cBox}

% Creates an environment for each type of theorem and assigns it a theorem text style from the "Theorem Styles" section above and a colored box from above
\newenvironment{theorem}{\begin{tBox}\begin{theoremeT}}{\end{theoremeT}\end{tBox}}
\newenvironment{exercise}{\begin{eBox}\begin{exerciseT}}{\hfill{\color{ocre}\tiny\ensuremath{\blacksquare}}\end{exerciseT}\end{eBox}}
\newenvironment{definition}{\begin{dBox}\begin{definitionT}}{\end{definitionT}\end{dBox}}
\newenvironment{example}{\begin{exampleT}}{\hfill{\tiny\ensuremath{\blacksquare}}\end{exampleT}}
\newenvironment{corollary}{\begin{cBox}\begin{corollaryT}}{\end{corollaryT}\end{cBox}}

%----------------------------------------------------------------------------------------
%	REMARK ENVIRONMENT
%----------------------------------------------------------------------------------------

\newenvironment{remark}{\par\vspace{10pt}\small % Vertical white space above the remark and smaller font size
\begin{list}{}{
\leftmargin=35pt % Indentation on the left
\rightmargin=25pt}\item\ignorespaces % Indentation on the right
\makebox[-2.5pt]{\begin{tikzpicture}[overlay]
\node[
draw=ocre!60,
line width=1pt,
circle,
fill=ocre!25,
font=\sffamily\bfseries,
inner sep=2pt,
outer sep=0pt] at (-15pt,0pt){\textcolor{ocre}{R}};\end{tikzpicture}} % Orange R in a circle
\advance\baselineskip -1pt}{\end{list}\vskip5pt} % Tighter line spacing and white space after remark

%----------------------------------------------------------------------------------------
%	SECTION NUMBERING IN THE MARGIN
%----------------------------------------------------------------------------------------

\makeatletter
\renewcommand{\@seccntformat}[1]{\llap{\textcolor{ocre}{\csname the#1\endcsname}\hspace{1em}}}
\renewcommand{\section}{\@startsection{section}{1}{\z@}
{-4ex \@plus -1ex \@minus -.4ex}
{1ex \@plus.2ex }
{\normalfont\large\sffamily\bfseries}}
\renewcommand{\subsection}{\@startsection {subsection}{2}{\z@}
{-3ex \@plus -0.1ex \@minus -.4ex}
{0.5ex \@plus.2ex }
{\normalfont\sffamily\bfseries}}
\renewcommand{\subsubsection}{\@startsection {subsubsection}{3}{\z@}
{-2ex \@plus -0.1ex \@minus -.2ex}
{.2ex \@plus.2ex }
{\normalfont\small\sffamily\bfseries}}
\renewcommand\paragraph{\@startsection{paragraph}{4}{\z@}
{-2ex \@plus-.2ex \@minus .2ex}
{.1ex}
{\normalfont\small\sffamily\bfseries}}

%------------------------------------------------------------------------------
% Hyperlinks
%------------------------------------------------------------------------------
\usepackage{nameref}
\usepackage{ifxetex,ifluatex}
\ifxetex
  \usepackage[setpagesize=false, % page size defined by xetex
              unicode=false, % unicode breaks when used with xetex
              xetex]{hyperref}
\else
  \usepackage[unicode=true]{hyperref}
\fi
\hypersetup{
%   backref=true,
%   pagebackref=true,
%   hyperindex=true,
  breaklinks=true,
  urlcolor= ocre,
%   bookmarks=true,
  bookmarksopen=false,
  pdfauthor={Marine Science Program, Dept Parks and Wildlife, WA},
  pdftitle={Annual Marine Protected Area Biodiversity Assets and Social Values Report},
  colorlinks=true,
  linkcolor=ocre,
  pdfborder={0 0 0}}

\urlstyle{same}  % don't use monospace font for urlstyle

%----------------------------------------------------------------------------------------
%  Page flow control
%----------------------------------------------------------------------------------------
\widowpenalty=10000
\clubpenalty=10000
\vbadness=1200
\hbadness=11000

%----------------------------------------------------------------------------------------
%	CHAPTER HEADINGS
%----------------------------------------------------------------------------------------

% The set-up below should be (sadly) manually adapted to the overall margin page septup
% controlled by the geometry package loaded in the main.tex document. It is possible to
% implement below the dimensions used in the goemetry package (top,bottom,left,right)... TO BE DONE

\newcommand{\thechapterimage}{}
\newcommand{\chapterimage}[1]{\renewcommand{\thechapterimage}{#1}}

% Numbered chapters with mini tableofcontents
\def\thechapter{\arabic{chapter}}
\def\@makechapterhead#1{
\thispagestyle{empty}
{\centering \normalfont\sffamily
\ifnum \c@secnumdepth >\m@ne
\if@mainmatter
\startcontents
\begin{tikzpicture}[remember picture,overlay]
\node at (current page.north west)
{\begin{tikzpicture}[remember picture,overlay]
\node[anchor=north west,inner sep=0pt] at (0,0) {
\includegraphics[width=\paperwidth,height=0.5\paperwidth]{\thechapterimage}};
%%%%%%%%%%%%%%%%%%%%%%%%%%%%%%%%%%%%%%%%%%%%%%%%%%%%%%%%%%%%%%%%%%%%%%%%%%%%%%%%%%%%%
% Commenting the 3 macros below removes the small contents box in the chapter heading

% Mini TOC background box
\fill[color=ocre!10!white,opacity=.2] (1cm,0) rectangle (
  3.5cm, % Mini TOC box width
  -3.5cm % Mini TOC box height
);

% Mini TOC text content
\node[anchor=north west] at (1.1cm,.35cm) {
  \parbox[t][8cm][t]{6.5cm}{
    \huge\bfseries\flushleft
    \printcontents{l}{1}{
    \setcounter{tocdepth}{1} % Mini TOC level depth
    }
  }
};

% Chapter heading
\draw[anchor=west] (5cm,-9cm) node [
rounded corners=20pt,
fill=ocre!10!white,
text opacity=1,
draw=ocre,
draw opacity=1,
line width=1.5pt,
fill opacity=.2,
inner sep=12pt]{
  \huge\sffamily\bfseries\textcolor{black}{
    \thechapter. #1\strut\makebox[22cm]{}
  }
};

%%%%%%%%%%%%%%%%%%%%%%%%%%%%%%%%%%%%%%%%%%%%%%%%%%%%%%%%%%%%%%%%%%%%%%%%%%%%%%%%%%%%%
\end{tikzpicture}};
\end{tikzpicture}}
\par\vspace*{210\p@} % was 230
\fi
\fi}

% Unnumbered chapters without mini tableofcontents (could be added though)
\def\@makeschapterhead#1{
\thispagestyle{empty}
{\centering \normalfont\sffamily
\ifnum \c@secnumdepth >\m@ne
\if@mainmatter
\begin{tikzpicture}[remember picture,overlay]
\node at (current page.north west)
{\begin{tikzpicture}[remember picture,overlay]
\node[anchor=north west,inner sep=0pt] at (0,0) {
  \includegraphics[width=\paperwidth,height=0.5\paperwidth]{\thechapterimage}};
\draw[anchor=west] (5cm,-9cm) node [rounded corners=20pt,
  fill=ocre!10!white,fill opacity=.6,inner sep=12pt,text opacity=1,
  draw=ocre,draw opacity=1,line width=1.5pt]{
  \huge\sffamily\bfseries\textcolor{black}{#1\strut\makebox[22cm]{}}};
\end{tikzpicture}};
\end{tikzpicture}}
\par\vspace*{210\p@}
\fi
\fi
}
\makeatother
%------------------------------------------------------------------------------
% End of Legrand's orange book header
%------------------------------------------------------------------------------

%------------------------------------------------------------------------------
% CKAN-o-Sweave specific header
%------------------------------------------------------------------------------
% A horizontal rule
\newcommand{\HRule}{\rule{\linewidth}{0.5mm}}

% lixltx2e provides \textsubscript
\usepackage{fixltx2e}

\usepackage{syntonly}
% \syntaxonly

%------------------------------------------------------------------------------
% Graphics
%------------------------------------------------------------------------------
% We will generate all images so they have a width \maxwidth. This means
% that they will get their normal width if they fit onto the page, but
% are scaled down if they would overflow the margins.
\makeatletter
\def\maxwidth{\ifdim\Gin@nat@width>\linewidth\linewidth\else\Gin@nat@width\fi}
\makeatother
\newcommand{\includeMPAgraphics}[1]{\includegraphics[width=0.8\linewidth]{#1}}
% \newcommand{\includeMPAgraphics}[1]{\includegraphics[width=\maxwidth]{#1}}

% Keep figures from floating to new page:
% http://tex.stackexchange.com/questions/2275/keeping-tables-figures-close-to-where-they-are-mentioned
% http://www.ctan.org/pkg/float
\usepackage{float}
\floatplacement{figure}{H}

% Allow transparent images
\usepackage{transparent}

% smaller figure captions
\usepackage[font=small,labelfont=bf]{caption}

%------------------------------------------------------------------------------
% Watermark
% Comment the next section in/out to toggle watermark
%------------------------------------------------------------------------------
\usepackage[firstpage]{draftwatermark} % add watermark to first page
\SetWatermarkText{Internal use only}
\SetWatermarkScale{0.3}
\SetWatermarkColor[gray]{0.85}
%
% %------------------------------------------------------------------------------
% % Common headings and text chunks
% %------------------------------------------------------------------------------
\newcommand{\ack}[3]{
Prepared by {#1} with input from {#2}\\
Management implications and overall assessment current as of {#3}; data current as marked.
}

\newcommand{\su}{
\section*{Summary}
\addcontentsline{toc}{section}{\textcolor{ocre}{Summary}}
}

\newcommand{\ass}{
\section*{Assessment}
\addcontentsline{toc}{section}{\textcolor{ocre}{Assessment}}
}

\newcommand{\co}{
\section*{Condition}
\addcontentsline{toc}{section}{\textcolor{ocre}{Condition}}
}

\newcommand{\pr}{
\section*{Pressure}
\addcontentsline{toc}{section}{\textcolor{ocre}{Pressure}}
}

\newcommand{\re}{
\section*{Response}
\addcontentsline{toc}{section}{\textcolor{ocre}{Response}}
}

\newcommand{\sy}{
\section*{Synopsis}
\addcontentsline{toc}{section}{\textcolor{ocre}{Synopsis}}
}

\newcommand{\mi}{
\section*{Management implications}
\addcontentsline{toc}{section}{\textcolor{ocre}{Management}}
}

\newcommand{\migen}{\subsubsection*{General}}
\newcommand{\miadm}{\subsubsection*{Management and administrative framework}}
\newcommand{\miedu}{\subsubsection*{Education and interpretation}}
\newcommand{\mipub}{\subsubsection*{Public participation}}
\newcommand{\mipat}{\subsubsection*{Patrol and enforcement}}
\newcommand{\mivis}{\subsubsection*{Management intervention and visitor infrastructure}}
\newcommand{\mires}{\subsubsection*{Research}}
\newcommand{\mimon}{\subsubsection*{Monitoring}}


\newcommand{\rf}{
\section*{References}
\addcontentsline{toc}{section}{\textcolor{ocre}{References}}
}

\newcommand{\references}{
\section*{References}
\addcontentsline{toc}{section}{\textcolor{ocre}{References}}
}

\newcommand{\sd}{
\subsubsection*{Source of data}
}

% \fg{figure caption as placeholder} creates a placeholder note
\newcommand{\fg}[1]{
\subsection*{Figure} {#1}
}

% \sh{report slug}{chapter slug}{section slug}{short TOC/header title}{Section heading}{Index keyword}
%
% creates:
% \section[section_short]{section_title}
% \sectionmark{section_short}
% \label{sec:report_slug-chapter_slug-section_slug}
% \index{index_keyword}
%
% To combine unnumbered subsections with short section titles, read
% http://tex.stackexchange.com/questions/168378/unnumbered-sections-and-section-marks
\newcommand{\sh}[6]{
% \subsection[#4]{#5} % Numbered with short version for TOC
\subsection*{#5} % Unnumbered without short version for TOC
\sectionmark{#4}
\addcontentsline{toc}{subsection}{\textcolor{ocre}#4}
\label{sec:#1-#2-#3}
\index{#6}
}

\newcommand{\sech}[6]{
\section*{#5} % Unnumbered without short version for TOC
\sectionmark{#4}
\addcontentsline{toc}{section}{\textcolor{ocre}#4}
\label{sec:#1-#2-#3}
\index{#6}
}

%\ch{i$url}{i$pth}{chapter_title}{{report_slug}{chapter_slug}{author}{contributors}{written}
\newcommand{\ch}[8]{
\write18{wget -x #1}
\chapterimage{#2}
\chapter{#3}
\label{chap:#4-#5}
\ack{#6}{#7}{#8}
}

%------------------------------------------------------------------------------
% CKAN integration
%------------------------------------------------------------------------------
% This section contains Latex macros to simplify using the output of R function
% "ckan_res" to produce subsections with text and figures retrieved from CKAN.


\newcommand{\mpa}[9]{
\write18{wget -x #1 } % Download image
% \subsection*{#2}
\begin{figure}
  \centering
  \includeMPAgraphics{#3}
  \caption{#4\\Source: #5\\\href{#6}{Data} last updated by {#7} on {#8}}
\end{figure}
}

%------------------------------------------------------------------------------
% CPR tables
%------------------------------------------------------------------------------
% This longer example demonstrates how to turn a repeating, complex Latex
% section into a compact, easy to use macro.
%
% Options for Condition Assessment:
% \cellcolor{green} EXCELLENT
% \cellcolor{green!25} GOOD
% \cellcolor{lightgray!25} SATISFACTORY
% \cellcolor{yellow!50} UNSATISFACTORY
% \cellcolor{red!50} POOR
%
% Options for Pressure Assessment:
% \cellcolor{green} LOW
% \cellcolor{lightgray!25} MODERATE
% \cellcolor{yellow!50} HIGH
%
% Options for Confidence (both Condition and Pressure)
% \cellcolor{green} HIGH
% \cellcolor{lightgray!25} MEDIUM
% \cellcolor{yellow!50} LOW
%
% Options for Pressure Trend
% \cellcolor{green} DECREASING
% \cellcolor{lightgray!25} CONSTANT
% \cellcolor{yellow!50} INCREASING

% CPR Table
% \cpr{+}{0}{-}{+}{-}{+}
% Positive = desirable, Negative = undesirable
% Argument 1: Condition Assessment: ++ + 0 - --
% Argument 2: Condition Confidence: + 0 -
% Argument 3: Pressure Assessment: + 0 -
% Argument 4: Pressure Confidence: + 0 -
% Argument 5: Pressure Trend: + 0 -
% Argument 6: Data ok / warning: + -
\newcommand{\cpr}[6]{%
  \ass
  \begin{center}
  \begin{tabular} {l l l} \\
  \hline
  CPR & Assessment & Confidence \\
  \hline
  Condition &
    \IfEqCase{#1}{%
      {++}{\cellcolor{green} EXCELLENT}%
      {+}{\cellcolor{green!25} GOOD}%
      {0}{\cellcolor{lightgray!25} SATISFACTORY}%
      {-}{\cellcolor{yellow!50} UNSATISFACTORY}%
      {--}{\cellcolor{red!50} POOR}%
      }[\PackageError{mpa-report}{Yikes, looks like the cpr macro had a fanny wobble. The 1st argument, Condition Status, can only be ++, +, 0, -, or --}{}] &
    \IfEqCase{#2}{%
      {+}{\cellcolor{green} HIGH}%
      {0}{\cellcolor{lightgray!25} MEDIUM}%
      {-}{\cellcolor{yellow!50} LOW}%
      }[\PackageError{mpa-report}{Invalid argument used in cpr macro. The 2nd argument, Condition Confidence, can only be +, 0, or -}{}]\\
  Pressure &
  \IfEqCase{#3}{%
      {+}{\cellcolor{green} LOW}%
      {0}{\cellcolor{lightgray!25} MODERATE}%
      {-}{\cellcolor{yellow!50} HIGH}%
      }[\PackageError{mpa-report}{Invalid argument used in cpr macro. The 3rd argument, Pressure Status, can only be +, 0, or -}{}] &
  \IfEqCase{#4}{%
      {+}{\cellcolor{green} HIGH}%
      {0}{\cellcolor{lightgray!25} MEDIUM}%
      {-}{\cellcolor{yellow!50} LOW}%
      }[\PackageError{mpa-report}{Invalid argument used in cpr macro. The 4th argument, Pressure Confidence, can only be +, 0, or -}{}]\\
  Pressure Trend &
  \IfEqCase{#5}{%
      {+}{\cellcolor{green} DECREASING}%
      {0}{\cellcolor{lightgray!25} CONSTANT}%
      {-}{\cellcolor{yellow!50} INCREASING}%
      }[\PackageError{mpa-report}{Invalid argument used in cpr macro. The 5th argument, Pressure Trend, can only be +, 0, or -}{}] &\\
  \hline
  \end{tabular}
  \end{center}
  \IfEqCase{#6}{
    {+}{}%
    {-}{\paragraph{Insuffient data warning} Little or no data was used in this assessment.}%
   }[\PackageError{mpa-report}{Incorrect argument used in cpr macro. The 6th argument, Data Warning, can only be +, or -! Notably, no 0 is allowed}{}]
}%


%------------------------------------------------------------------------------%
% CPR overview page macros
%
\newcommand{\ovpre}{
\begin{table}[h]
\centering
\caption{Condition-Pressure Summary. Confidence in assessments is expressed
through symbols: *** High Confidence, ** Medium Confidence, * Low Confidence in
assessment. Assessments based on little or no data are marked as "insufficient".}
\label{cpr-summary}
\begin{tabular}{lllll}
\hline
Asset & Condition & Pressure & Trend & Data \\ \hline
}

\newcommand{\ovpost}{
\hline
\end{tabular}
\end{table}
}

\newcommand{\ovrow}[7]{
{#7} &
\IfEqCase{#1}{%
      {++}{\cellcolor{green} Excellent}%
      {+}{\cellcolor{green!25} Good}%
      {0}{\cellcolor{lightgray!25} Satisfactory}%
      {-}{\cellcolor{yellow!50} Unsatisfactory}%
      {--}{\cellcolor{red!50} Poor}%
      }[\PackageError{mpa-report}{Yikes, looks like the ovrow macro for #7 had a fanny wobble.
      The 1st argument, Condition Status, can only be ++, +, 0, -, or --}{}]
\IfEqCase{#2}{%
      {+}{***}%
      {0}{**}%
      {-}{*}%
      }[\PackageError{mpa-report}{Invalid argument used in ovrow macro for #7. The 2nd argument,
      Condition Confidence, can only be +, 0, or -}{}] &
\IfEqCase{#3}{%
      {+}{\cellcolor{green} Low}%
      {0}{\cellcolor{lightgray!25} Moderate}%
      {-}{\cellcolor{yellow!50} High}%
      }[\PackageError{mpa-report}{Invalid argument used in ovrow macro for #7. The 3rd argument,
      Pressure Status, can only be +, 0, or -}{}]
\IfEqCase{#4}{%
      {+}{***}%
      {0}{**}%
      {-}{*}%
      }[\PackageError{mpa-report}{Invalid argument used in ovrow macro for #7. The 4th argument,
      Pressure Confidence, can only be +, 0, or -}{}] &
\IfEqCase{#5}{%
      {+}{\cellcolor{green} Decreasing}%
      {0}{\cellcolor{lightgray!25} Constant}%
      {-}{\cellcolor{yellow!50} Increasing}%
      }[\PackageError{mpa-report}{Invalid argument used in ovrow macro for #7. The 5th argument,
      Pressure Trend, can only be +, 0, or -}{}] &
\IfEqCase{#6}{
      {+}{}%
      {-}{Insufficient}%
      }[\PackageError{mpa-report}{Incorrect argument used in ovrow macro for #7. The 6th argument,
      Data Warning, can only be +, or -! Notably, no 0 is allowed}{}]\\
}


%-------------------------------------------------------------------------------
% Assessment (from 2015)
%
\newcommand{\asspre}{
\ass
\begin{table}[h]
\centering
\caption{Condition-Pressure Summary. Confidence in assessments is expressed
through symbols: *** High Confidence, ** Medium Confidence, * Low Confidence in
assessment. Assessments based on little or no data are marked as "insufficient".}
\label{cpr-summary}
\begin{tabular}{llll}
\hline
% Park & Asset &
Year & Condition & Pressure & Data \\ \hline
}

\newcommand{\asspost}{
\hline
\end{tabular}
\end{table}
}

\newcommand{\assrow}[8]{
% {#1} &
% {#2} &
{#3} &
\IfEqCase{#4}{%
      {++}{\cellcolor{green} Excellent}%
      {+}{\cellcolor{green!25} Good}%
      {0}{\cellcolor{lightgray!25} Satisfactory}%
      {-}{\cellcolor{yellow!50} Unsatisfactory}%
      {--}{\cellcolor{red!50} Poor}%
      }[\PackageError{mpa-report}{Macro assrow, #1 #2 #3: The 4th argument,
      Condition Status, can only be ++, +, 0, -, or --}{}]
\IfEqCase{#5}{%
      {+}{***}%
      {0}{**}%
      {-}{*}%
      }[\PackageError{mpa-report}{Macro assrow, #1 #2 #3: The 5th argument,
      Condition Confidence, can only be +, 0, or -}{}] &
\IfEqCase{#6}{%
      {+}{\cellcolor{green} Low}%
      {0}{\cellcolor{lightgray!25} Moderate}%
      {-}{\cellcolor{yellow!50} High}%
      }[\PackageError{mpa-report}{Macro assrow, #1 #2 #3: The 6th argument,
      Pressure Status, can only be +, 0, or -}{}]
\IfEqCase{#7}{%
      {+}{***}%
      {0}{**}%
      {-}{*}%
      }[\PackageError{mpa-report}{Macro assrow, #1 #2 #3: The 7th argument,
      Pressure Confidence, can only be +, 0, or -}{}] &
\IfEqCase{#8}{
      {+}{}%
      {-}{Insufficient}%
      }[\PackageError{mpa-report}{Macro assrow, #1 #2 #3: The 8th argument,
      Data Warning, can only be +, or -! Notably, no 0 is allowed}{}]\\
}

