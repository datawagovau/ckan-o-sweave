This chapter gives an overview of the authors' work flow.

\subsection*{Overview}

\begin{itemize}
	\item[Set-up]
		One-off report automation environment set-up: Bitbucket, RStudio,
		Data Catalog, Slack chat
	\item[Git pull]
		Open \href{https://rstudio.dpaw.wa.gov.au/}{RStudio} and "Git pull" to
		get the latest revision of the source
	\item[Compile PDF]
		Open a report from \texttt{reports/NUMBER\_PARK.Rnw} and Compile PDF
	\item[Edit source]
		Ctrl-click a section in the PDF you wish to open the source file in RStudio.
		See next chapter for detailed instructions.
	\item[Review changes]
		Compile the report to PDF again (not the chapter) to review changes and
		fix errors if necessary.
	\item[Commit and push]
		When you have finished a chapter and the report compiles fine,
		stage-commit-pull-push your local changes to the source code into the
		central "master" repository.
\end{itemize}

\subsection*{Set-up: Bitbucket}
Authors must have a free \href{https://bitbucket.org/}{Bitbucket} account, and
the report maintainer must add these bitbucket accounts with "write" permissions to
the \href{https://bitbucket.org/dpaw/mpa-reporting/}{Bitbucket repository}.

\subsection*{Setup: RStudio Server}
Authors of this report use RStudio Server, running on \\
\texttt{kens-xenmate-dev.corporateict.domain}.
Authors need to open a Remote Desktop Connection into this server (with their
Departmental account) in order to have a local account and a home directory
created. RStudio Server accepts Departmental logins, but requires a home
directory to already exist.

Authors need to configure their \href{https://rstudio.dpaw.wa.gov.au/}{RStudio Server}
as follows under Tools/Global Options:

\begin{itemize}
	\item[Sweave] Enable shell escape commands, insert numbered sections and subsections
	\item[Git/SVN] Create RSA key... (Create without passphrase), then "View public key" and copy key.
\end{itemize}

In \href{https://bitbucket.org/}{Bitbucket}: top tight > Manage account > SSHkeys >
Add key > paste key and name it \texttt{yourusername@kens-xenmate-dev}.

In Tools/Shell, user name and email are set as:
\begin{verbatim}
\$ git config --global user.name "John Doe"
\$ git config --global user.email johndoe\@example.com
\end{verbatim}

In File > New Project > Version Control > Git > clone URL: \\
\texttt{git@bitbucket.org:dpaw/mpa-reporting.git},
create as subdirectory of (create if not existing) ~/projects.

Run the script \texttt{scripts/installation.R} to download required R packages.

Test the installation by compiling one report.Rnw - a PDF of the report should open.

\subsection*{Version control: Git}
This report is compiled from source documents within a folder structure.
The source is under version control, with one "master" repository living on
\href{https://bitbucket.org/dpaw/mpa-reporting/}{Bitbucket} and full copies of
the complete source on each author's machine.

\subsubsection*{Git pull}
Git pull receives the latest revisions from the master repository (the sum of all other authors'
contributions since your last update). If no new revisions are available, git pull will exit
with a "already up to date" message.

\subsubsection*{Git stage and commit}
If a file is changed, it has to be staged (marked for addition by ticking the check box in
RStudio's Git facet) and committed (added) to a new revision, which acts like a local restore point.

\subsubsection*{Git push}
Git push uploads all new local revisions to the master repository. The master repository must
not have any newer revisions than the local repository - otherwise, a git push will fail
harmlessly with an error message.

Hence, local changes are typically staged, committed, then eventual remote revisions are pulled,
before a final push sends the local revisions to the master.

The following chapter will explain the technical details of writing this report.
