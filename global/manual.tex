This chapter is targetted towards authors and contains technical details required to contribute content.
Authors receive an introduction and setup of their report writing environment.
To subsequently contribute content, a basic understanding of the gritty innards
of the reports is highly desirable for the authors.

The content of the reports lives in two main locations:
The data catalog holds data sets, figures, captions, and metadata around figures,
while the reports hold summary, assessment, figure titles, synopsis, management
implications, and references (which are moved to a separate chapter by the editors).

\section*{To R is human, to Sweave - divine}
Chapters are \href{http://www.stat.uni-muenchen.de/~leisch/Sweave/}{Sweave} documents,
capable of running R code embedded in the \LaTeX\ markup.
Document content is split up over separate files at the chapter level for easier handling.

Across the Sweave documents, chunks of R code are used in three ways:
\begin{itemize}
\item To define internal variables, separating the document structure
	(defined in \LaTeX) from document content (defined as R variables),
\item to run custom-made R functions, e.g. to use R to retrieve content from online
	sources, such as the data catalog, and
\item to run arbitrary R code for real-time processing and analysis of data.
\end{itemize}

\subsection*{Defining internal variables with R}
While the text content of report, chapter and section headings will differ,
they share the exact same structure. To separate out structure from content, the
structure is defined in \LaTeX\ markup, while the content is defined
in R chunks, and shared with the \LaTeX\ side through R variables.

\section*{Chapter Structure}
Each chapter consists of the same sections: Summary, Assessment, Figures
(under the subheadings of Condition, Pressure, and Response Indicators),
a general Synopsis, Management Implications with several subheadings,
and References.

\subsection*{Chapter heading}
In the chapter heading, adjust the variables to reflect authorship and your input.
{\small\begin{verbatim}
chapter\_slug $<-$ "dugong"
chapter\_title $<-$ "Dugong"
author $<-$ "Kevin Bancroft"
contributors $<-$ "Kim Friedman, Andrew Halford"
written $<-$ "August 2014"
revised $<-$ ""
# Kim Friedman 2015-08-04 done
i <- ckan_res("6809b734-366d-4a4a-86b5-15d77ad87a28")
\end{verbatim}}

Adjust the variables as needed (mainly author, contributors, written):
\begin{itemize}
\item{chapter\_slug} A short, url-safe label for the chapter for internal cross-reference.
This will only ever ben used internally, and never be seen in the final PDF.
\item{chapter\_title} The chapter title as displayed in the final PDF.
\item{author} The chapter author (asset leader) as displayed in the final PDF.
\item{contributors} Contributors of data, figures, and/or text as displayed in the
final PDF.
\item{written} The date on which the whole chapter was submitted for editorial review.
Setting this date signals that all author and contributor input has been provided.
\item{revised} Once the chapter has been revised, the date will be set to indicate
that the content of this chapter is finalised, and the chapter is ready for final
editing touches.
\item{comment} Individual authors are encouraged to leave comments about their contribution status.
The example shows a comment by someone called Kim Friedman indicating that his
contributions have been completed.
\item{i} Insert the resource ID of the title image here.
\item{macro \textbackslash ch} This macro generates the chapter heading from the
variables given above.
\end{itemize}

\subsection*{Summary}
Under the macro \texttt{\textbackslash su}, which generates the heading "Summary",
authors are to write a general Summary of the asset's status.

\subsection*{Overall Assessment}
Authors only have to set the arguments 4 to 8 of the assessment macro to represent
condition and pressure assessment and confidence, as well as data warning as
explained in the help text.

From 2016, in each new reporting cycle, the last line (containing \texttt{mpa\_period})
has to be copied, the \texttt{mpa\_period} has to be replaced with the last period
(e.g. 2015-2016), and the arguments 4 to 8 have to be adjusted to reflect the current
assessment.

{\small\begin{verbatim}
%------------------------------------------------------------------------------
% CPR Assessment
% \cpr{+}{0}{-}{+}{-}{+}
% Positive = desirable, Negative = undesirable
% report_title: The park name
% chapter_title: The asset name
% mpa_period: The FY of this assessment
% Argument 4: Condition Assessment: ++ + 0 - --
% Argument 5: Condition Confidence: + 0 -
% Argument 6: Pressure Assessment: + 0 -
% Argument 7: Pressure Confidence: + 0 -
% Argument 8: Data ok / warning: + -
\asspre
\assrow{\Sexpr{report_title}}{\Sexpr{chapter_title}}{2013--2014}{+}{0}{+}{0}{+}
\assrow{\Sexpr{report_title}}{\Sexpr{chapter_title}}{\Sexpr{mpa_period}}{+}{0}{+}{0}{+}
\asspost
\end{verbatim}}

\subsection*{Figures}
Underneath the macros for Condition (\texttt{\textbackslash co}),
Pressure (\texttt{\textbackslash pr}), and Response (\texttt{\textbackslash re})
indicators, repeating R chunks represent individual figures:

{\small\begin{verbatim}
%------------------------------------------------------------------------------
% INDICATOR Habitat
<<echo=FALSE>>=
section_slug <- "habitat" # Section label
section_title <- "Habitat" # Full section heading
section_short <- "Habitat" # TOC and page header
keyword <- "Habitat"
g <- ckan_res("ccc68eb7-8105-4cc8-8112-57bf1558e82f")
@
\sh{\Sexpr{report_slug}}{\Sexpr{chapter_slug}}...
\mpa{\Sexpr{g$url}}{\Sexpr{g$ind}}{\Sexpr{g$pth}}...
\end{verbatim}}

To remove a figure, the section heading (\texttt{\textbackslash sh}) and the
figure (\texttt{\textbackslash mpa}) macros can be commented out.
This retains the code for later use or re-use, while hiding it from the PDF output.
Commenting out the variable \texttt{g} prevents loading of an unused resource and
shaves off two seconds of compiling the PDF.

To add a figure, the whole block can be copied, and values in the R chunk adjusted:

\begin{itemize}
  \item{section\_slug} The slug must be url-safe and should be unique within the chapter.
  It is used internally to create cross-references, but will never be seen in the
  PDF output.
  \item{section\_title} The section heading serves as figure title. It should be
  comprehensive, yet avoid unneccessary verbosity.
  \item{section\_short} The short version of the section title will be used in
  page headers and the table of contents. It should honour their length limits.
  \item{keyword} The keyword(s) will be listed in the Index at the end of the
  PDF document. Avoid duplicates. Captitalise words.
  \item{g} Paste the figure's resource ID from the data catalogue CKAN into
  \texttt{ckan\_res}.
\end{itemize}

\subsection*{Synopsis}
Authors are to write the Synopsis for the whole asset update here.
Where time permitted, individual synopses from previous years have been moved
here for authors' convenience, and have to be merged and shortened.

\subsection*{Management Implications}
Authors are to write management implications underneath the appropriate subheadings.
Where time permitted, individual managment implications from previous years,
excluding copy-pasted phrases, have been moved underneath
"Management Implications - General" for authors' convenience, and have to be
shortened and distributed into the appropriate subheadings.

\subsection*{References}
At the end of each chapter, a section for references exists to paste in all cited literature.
After author contributions have been provided, the editors merge each report's
references into one single chapter.

\subsection*{Special characters}
To typeset the following characters, see the source code of this section.
Some special characters (\& \% \$ \# \_ \{ \} \@) have to be backwhacked
(prepended with a backslash) in \LaTeX. Some other special characters (\textasciitilde,
\textasciicircum, and \textbackslash)have their own \LaTeX\ command.
Formulas like $H^{2}O$ and $NH_{4}^{+}$ need to enclosed in dollar signs.
More \href{http://en.wikibooks.org/wiki/LaTeX/Mathematics}{here}.

\subsection*{Performance}
The slowest step in the authors' workflow is the compilation of the PDF.
As authors will typically be interested in only one chapter -- the one they currently work on --
the other chapters can be excluded temporarily by commenting out as follows:

{\small\begin{verbatim}
% \SweaveInput{assets/pinnipeds/nkmp-pinnipeds.Rnw}
\SweaveInput{assets/finfish/nkmp-finfish.Rnw}
\end{verbatim}}

In this example, the Pinniped chapter is deactivated, while the Finfish chapter
will be compiled, and all of its figures will be downloaded freshly from the
data catalogue.

\begin{remark}
RStudio shortcut: CTRL-c toggles comments, e.g. to temporarily deactivate chapters.
Keep Water Quality and Common Indicators active, as all other chapters crossreference them.
It is not necessary to commit the changed report .Rnw files.
\end{remark}


\subsection*{References and useful links}
\href{https://support.rstudio.com/hc/en-us}{RStudio knowledge base}\\
\href{http://en.wikibooks.org/wiki/LaTeX}{The \LaTeX\ book}\\
\href{http://tex.stackexchange.com/}{TeX Stackexchange Forum}\\
